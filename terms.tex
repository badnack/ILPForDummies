\section{Terms}
\label{sec:terms}
In order to better understand the content of this document, in this
section is presented a little summary of the terms that will be used from now on together with
their explanations.

\begin{itemize}
  \item Formula: in first-order logic a formula can be thought as a sequence of symbols
    (i.e., variables and connectives) for which it makes sense to ask, under a certain
    interpretation -- i.e., substituting free variables with real values --, ``is it true?''
  \item Fact: a fact is a ground atomic formula.  A formula is said to be atomic when
    it does not contain any logical connectives (e.g., a predicate symbol together with its arguments).
    A ground atomic formula is an atomic formula which does not contain
    any free variables. For example, the fact ``Ada is John's mother'' can be formally expressed
    by: $mother(Ada, John)$. In this case $mother$ is a predicate and Ada and John are its two
    arguments.
  \item Rule: Expression that becomes true if the premise expression is true.
  \item Rulebase: collection of fact and rules.
  \item Horn clause: expression of the form $B \leftarrow B_1,B_2,...,B_k$ where $k \in \mathbb{I}$ and
    B and each $B_i$ is atomic.
  \item Embedded implication: expression of the form $B \leftarrow \phi_1,\phi_2,...,\phi_k$
    where $k \in \mathbb{I}$, B is atomic and each $\phi_i$ is a Horn clause.
    Note that embedded implications include Horn clauses as a special case, and Horn
    clauses include atomic formulas as a special case.
  \item Logic Semantic: study of the meaning of a formal logic. That is to say the study
    of the meaning of the logical connectives, how formulas can be combined to infer new formulas,
    how formulas are constructed and how they are related.
    Logic semantic may be studied relying on different approaches. One of the most
    used relies on mathematical models (Model-theoretic semantics).
  \item Logic syntax: rules that regulate how the symbols allowed in a certain type
    of formal logic are combined.
  \item Logic Programming:  programming paradigm based on formal logic.
  \item Prolog: a declarative logic programming language.
\end{itemize}
