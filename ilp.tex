\section{Intuitionistic Logic Programming}
\label{sec:ilp}
One of the best definitions about logic programming can be found on Wikipedia:
``Logic programming is a type of programming paradigm which is based on formal logic.
A program written in a logic programming language merely is a set of sentences in
logical form, expressing facts and rules about some problem domain''.

Intuitionistic Programming Language means that the program is written using
intuitionistic semantic and syntax. As often happens, only a fragment of
the considered logic is implemented, not the entire and complete logic.
For example, Prolog doesn't allow disjunctive facts or conclusions, as obviously
classical logic does. The choice to use only a frame of the considered logic is usually made
to make the employed interpreter more efficient. An example of an interpreter for
intuitionistic logic programming is shown in Section~\ref{sec:apifhr}.

The easiest way to implement hypothetical reasoning is to temporarily modify the rule
base, try to infer the query and eventually restore the rulebase to the original content.

Let's consider again the example in Section~\ref{sec:formalization}. In order
to give a response to the query we need to temporarily introduce the rule
$worked\_on(Phil, Icarus)$ within the rulebase R, try to infer $wall\_of\_fame(Phil)$
and eventually remove the rule for the rulebase.

This is exactly how hypothetical queries are validated in intuitionistic logic
programming.

\section{A Prolog Interpreter for Hypothetical Reasoning}
\label{sec:apifhr}
We said in Section~\ref{sec:alfhr}, that hypothetical reasoning has a intuitionistic semantic.
Therefore what we need merely is an intuitionistic interpreter.

The implementation of a Prolog interpreter for intuitionistic logic,
has been studied by L. Thorne McCarty and Leon A. Shklar in~\cite{McCartyS94}.

This work presents a ``syntactically restricted language'' (as they call it), which basically
is a language which implements ``part'' of the intuitionistic logic.
As they demonstrated in their work, this ``syntactically restricted language'' is
expressively equivalent to the full language, though.

Particularly, they decomposed intuitionistic logic into two restricted subsets:
\begin{itemize}
  \item the class of embedded implications: that is to say a class of formulas as
    \begin{equation}
      B(x) \leftarrow A_1(x) \wedge A_2(x) \wedge ... \wedge A_n(x),\ n \in \mathbb{I}
    \end{equation}
    where B(x) is an atomic formula and each $A_i(x)$ is either an atomic formula
    or another embedded implication with (zero or more) embedded universal quantifiers
    (i.e., $\forall$). Note that here L. Thorne McCarty and Leon A. Shklar relaxed
    a little bit the definition of embedded implication. In fact, differently from
    what has been said in Section~\ref{sec:terms}, here embedded implications allow
    other embedded implications in their body. Note that this definition is more 
    general. In fact, as said in Section~\ref{sec:terms}, embedded implications
    include Horn clauses as special case.

  \item the class of disjunctive and existential assertions: that is to say a class
    of formulas like:
    \begin{equation}
      B(x) \rightarrow \wedge^{n}_{i=1} Q_i(x)\ and\ B(x) \rightarrow (\exists y) \wedge^m_{j=1}Q_j(x;y)
    \end{equation}
    Where the x variables are universally quantified and no x variables are allowed
    on the right-hand side which do not appear in B(x). Each Q is an atomic formula.
\end{itemize}

Using these two classes, the following implication rules are defined by McCarty
and Shklar:
\begin{enumerate}[I)]
  \item $R \vdash A$
    \begin{equation}
      \begin{split}
        &if\ there\ is\ a\ rule\ B(x) \leftarrow \wedge^{n}_{i=1} A_i(x)\ in\ R\ and\ a\ ground\ substitution\ \theta\ for\ x:\\
        &A=B(x)\theta\ and\ R \vdash A_i(x)\theta,\ for\ i = 1...n
      \end{split}
    \end{equation}

    \item $R \vdash A \leftarrow \wedge^{n}_{i=1} Q_i(x)$
      \begin{equation}
        if\ R \cup^n_{i=1}\{Q_i(x)\} \vdash A
      \end{equation}

    \item $R \vdash (\forall y) A(y)$
      \begin{equation}
      \begin{split}
        &if\ R \vdash A(c)\ for\ some\ tuple\ of\ constants\ c=\langle c_1,c_2,...,c_k\rangle\ that\\
        &do\ not\ appear\ in\ R\ or A(y)
      \end{split}
    \end{equation}

    \item $R \vdash A$
      \begin{equation}
      \begin{split}
        &if\ there\ is\ a\ rule\ B(x) \rightarrow \vee^n_{i=1}Q_i(x)\ in\ R\ and\ a\ ground\ substitution\ \theta\ for x:\\
        &R \vdash B(x)\theta\ and\ R \cup \{Q_i(x)\theta\}\vdash A,\ for\ i=1,...,n
      \end{split}
    \end{equation}

    \item $R \vdash A$
      \begin{equation}
      \begin{split}
        &if\ there\ is\ a\ rule\ B(x) \rightarrow (\exists y)\wedge^m_{j=1}Q_j(x;y)\ in\ R\ and\ a\ ground\ substitution\ \theta\ \\
        &for\ x\ such\ that\ R\ \vdash B(x)\theta\ and:\\
        &R \cup (\cup^m_{j=1}\{Q_j(x;c)\theta\})\vdash A for\ some\ tuple\ of\ constants\ c\ \langle c_1,c_2,...,c_k \rangle\ that\ \ do\\
        &not\ appear\ in\ R\ or\ A
      \end{split}
    \end{equation}
\end{enumerate}

Note that these rules are not different from those presented in Section~\ref{sec:rfhi}
(the logic is the same after all). They have been just rewritten using the two classes
or rules that McCarty and Shklar defined.

The interpreter is conceptually divided in two different parts, one for each class of
rules: embedded implications and disjunctive and existential assertions.

In order to understand the next section a little knowledge of Prolog is required.
Most of the concepts will be explained anyway. For a good overview of Prolog
please refer to~\cite{Prolog1},~\cite{Prolog2} and~\cite{Prolog3}.

\subsection{Embedded Implications Interpreter}
\label{sec:eii}

The interpreter for embedded implications is reported in the listing below.

\begin{lstlisting}
  prove1(true,_,_,_) :- !.
  prove1((G1,G2),HypDB,Vars,N) :- !,prove1(G1,HypDB,Vars,N),
      prove1(G2,HypDB,Vars,N).
  prove1((Goal:-Hyp),HypDB,Vars,N) :-
      append(Hyp,HypDB,NewHypDB),N1 is N+1,
      add_variables((Goal:-Hyp),N,Vars,Vars1),
      !,prove1(Goal,NewHypDB,Vars1,N1).
  prove1(forall(Ys,(Goal:-Hyp)),HypDB,Vars,N) :-
      append(Hyp,HypDB,NewHypDB),N1 is N+1,
      generate_constants(Ys,N1),
      add_variables((Goal:-Hyp),N,Vars,Vars1),
      !,prove1(Goal,NewHypDB,Vars1,N1).
  prove1(Goal,HypDB,Vars,_) :- member(Goal,HypDB),\+violation(Vars).
  prove1(Goal,HypDB,Vars,N) :- clause(Goal,Body),\+violation(Vars),
  prove1(Body,HypDB,Vars,N).
\end{lstlisting}

The interpreter is not so easy as one may hope. In this document a simple
intuitive explanation will be given.

The first clause is trivial: if the premise of the embedded implication is equal to True,
the rule is satisfied. The second clause succeeds if the conjunction of the two embedded implications
G1 and G2 is proved. The third clause works as follow: to check if ``Goal''
follows from $Hyp$, $Goal$ is appended to the current $HypDB$ and it will be tried to
prove that $Goal$ follows from $NewHypDB$. Such a test is implemented by the
fifth clause. The fourth clause follows the
same approach but handles embedded implications containing universal quantifiers
(i.e., $\forall$). For a detailed overview of this clause, please refer to~\cite{McCartyS94}.
Finally the sixth and last clause is trivial as well: to prove that a goal follows
from a set of hypothetical rules (i.e., $HypDB$) it has to be proved its body. Note
that the fact that $Body$ is the actual body of $Goal$ is true if the
predicate $clause(Body,\ Goal)$ is true as well.


\subsection{Disjunctive Assertion Interpreter}
\label{sec:dai}
The interpreter for disjunctive assertions is reported in the listing below.

\begin{lstlisting}
  prove2([Ans/Split<-[]],DB,Suspend,Stack) :-
      (dsplit(DB,Split,Suspend,Stack);!,extend(DB,Suspend,Stack)).
  prove2([Ans/Cases<-[],Frame|Pnode],DB,Suspend,Stack) :-
      copy_term(Frame,A/OldCases<-[Call|Goals]), Call=Ans,
      append(OldCases,Cases,NewCases),
      !,prove2([A/NewCases<-Goals|Pnode],DB,Suspend,Stack).
  prove2(Node,DB,Suspend,Stack) :- Node = [_<-[Goal|_]|_],
      findall(Goal/[]<-[],db(Goal,DB),Fr1),
      findall(Goal/[]<-Body,rule(Goal,Body),Fr2),
      findall(Goal/Cases<-Body,drule(Goal,Cases,Body),Fr3),
      append(Fr2,Fr3,Fr),append(Fr1,Fr,Frames),
      suspend(Goal,Node,Suspend,NewSuspend),
      !,extend(DB,NewSuspend,[branch(Frames,Node)|Stack]).
  dsplit(DB,[],Suspend,Stack) :- !.
  dsplit(DB,[A|Split],Suspend,Stack) :-
      suspensions(A,Suspend,NewSuspend,Branches),
      append(Branches,Stack,NewStack),
      !,extend([A|DB],NewSuspend,NewStack),
      dsplit(DB,Split,Suspend,Stack).
\end{lstlisting}

Since the use of disjunctive assertions, the approach followed here is to ``restart'' the
proof procedure several times (as many as the number or disjunctive assertions).

The approach followed by this interpreter is well explained in the work~\cite{McCartyS94}.

``Backtracking information (i.e., ``how to restart a proof'') is maintained explicitly in the variable Stack.
Stack is a list of branches in the form $branch(Frames,Node)$, where Frames and Node are both
lists of frames. A frame has the form $Ans \leftarrow Goals$, where Ans is an atom and Goals is a list of
atoms. A call to findall collects a list of frames whose Ans atom
would unify with the first goal -- the calling goal -- in the first frame of the current Node. These
frames are then packaged into a new branch, which is pushed onto the current Stack and processed
by extend. When the list of goals is empty, the second clause unifies Ans with Call and
begins processing the next frame in the current Node, minus the calling goal. Finally, if the list
of unprocessed frames is empty, the first clause writes out the answer and calls extend
to process the stack again''~\cite{McCartyS94}.

The above lines do not explain in detail the previous listing, like we did in
Section~\ref{sec:eii}, but they should give to the reader a good overview of how this
interpreter works.
