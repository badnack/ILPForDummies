\begin{appendices}
  \section{Proof and Model theory}
  Intuitionistic logic, like most logics, can be defined in two formal ways:
  model-theoretically or proof-theoretically.
  These approaches merely are two ways of formalizing systems so that
  their behavior can be studied.

  Particularly, proof theory is a branch of mathematical logic that represents proofs (i.e.,
  deductive argument for a mathematical statement) as formal
  mathematical objects, so to apply mathematical techniques in
  order to study them and to derive new conclusions.

  In a logic context, a proof can be seen as a program (as stated by~\cite{PropAsTypes}).
  This relationship is known as Curry-Howard correspondence. Since there exists
  such a direct relationship between logic programs and proofs, it is
  natural that proof theory can be employed to model and study a logic programming
  language. Specifically, in a logic as programming language perspective,
  proof-theory is used to define the rules of execution and how they work.

  Summarizing, we can say that a proof-theoretic approach shows in details how certain
  properties of a system are satisfied.

  Another way to model and study the meaning (i.e., the semantic) of a logic is through model
  theory. As briefly mentioned in this very write up, the intuitionistic semantic can be
  modeled by using the Kripke model. A mathematical model merely is the description of
  a system using mathematical concepts and language. This means that logics
  (as for example intuitionistic logic) can be studied by formalizing them as
  mathematical systems.

  Differently from a proof-theoretic perspective, a model describes a system from a
  higher point of view. In a logic programming context this means that it describes
  programs in terms of their inputs (which can be abstract), their outputs
  and (generally) a set of internal states, which in a programming language context
  represent different rulebases of the system.

  Given a logic system (e.g., an intuitionistic logic interpreter), properties and
  arguments can be proved by modeling such a system employing a proof-theoretic approach
  as well as a model-theoretic approach.
  There is not a ``right'' or ``wrong'' approach: according on what properties we want to prove
  an approach may merely be more complex and difficult to use than the other.
\end{appendices}
