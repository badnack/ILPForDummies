\section{A Logic For Hypothetical Reasoning}
\label{sec:lfhr}
In literature there are several works that explore the advantages and the potentiality
to express hypothetical reasoning in formal logic.
For example, Gabbay in~\cite{Gabbay1984319} stated that if Prolog
were augmented with hypothetical reasoning,
it would be possible to formally express the British
Nationality Act (rules such as
``You are eligible for citizenship if your father would be eligible if he were
still alive'' are present).

As can be evinced by this famous example, expressing hypothetical reasoning in a given logic
(e.g., intuitionistic) merely means to introduce ``if'' concept in such a logic.
In other words expressions like
``if we assume that the formula A were valid then the formula B would be valid as well'' could be
formalized.

Let's see a practical (and easier) example. Let's consider a set of rules and regulation of a
software company. Precisely, let's consider a rulebase composed by facts like
$worked\_on(Marcus, Karonte)$ and $best\_project(Katrina, 2012)$ and rules like:

\begin{equation}
  wall\_of\_fame(e) \leftarrow \exists p [worked\_on(e, p), has\_been\_prized(p)]
\end{equation}

The former fact means that Marcus during his employment within the software company has worked
on a project named ``Karonte'', the latter fact means that a project named ``Katrina'' was
prized as ``best project'' in 2012 and finally the rule means
that the name of the employee $e$ is eligible to appear on a ``wall of fame'' if
he worked at least on a project which has been prized as ``best project'',
sometime in the past.

Let's assume now that we want to answer the following question: ``Would
be Phil eligible to appear on the wall of fame if he worked on the project Icarus?''.

A little more formally this \textit{query} means: ``if the fact $worked\_on(Phil, Icarus)$ were
added to the rulebase, could $wall\_of\_fame(Phil)$ be inferred?

The example above is a very simple, but it should be enough to show what we mean
with the expression ``Hypothetical reasoning'': the ability to construct queries as well as
rules which are based on hypothetical facts.

How these queries and rules have been formalized will be treated in
Section~\ref{sec:formalization}.

\section{Intuitionistic Logic}
\label{sec:il}
Before starting to talk about how hypothesis can be represented in a formal way,
let's talk briefly about intuitionistic logic.

Intuitionistic logic is a type of formal logic.
According Wikipiedia ``Intuitionistic logic is a system of symbolic logic that
differs from classical logic by replacing the traditional concept of truth with
the concept of constructive provability''.

This means that in intuitionistic logic a statement (e.g., a proposition) to be true
has to be formally proven.
Classically (i.e. in classical logic) a value of truth (i.e,
true or false) is assigned to each proposition, regardless of such a proposition
has been proven or not. This is not the case in intuitionistic logic. Furthermore
many classically valid tautologies are not theorems of intuitionistic logic (e.g.,
law of excluded middle), but all theorems of intuitionistic logic are valid classically.
For this very reason intuitionistic logic can be seen as a weak form of classical logic.

The syntax of intuitionistic logic is similar to the syntax of classical logic,
even though, the meaning of some of its connectives is not classical.

As example, let's consider the ``double negation elimination axiom''.
In classical logic we know that $P \to \lnot\lnot P$ and $\lnot\lnot P \to  P$, where P
is a proposition. In intuitionistic logic only the first axiom is valid.

This is merely because the notion of ``negation'' in intuitionistic logic is not classical.
While classically $\lnot P$ means that P is False, Intuitionistically $\lnot P$ means that
there exists a demonstration which proves the inexistence of a demonstration of P.
Said this, if P is demonstrable it is impossible
to prove that there is not exist a demonstration of P (i.e.,  $P \to \lnot\lnot P$).

Vice versa, if no demonstrations that prove that ``there is not any demonstrations of P'' are available,
it is not possible to conclude that a demonstration of P exists (i.e., $\lnot\lnot P \to  P$ is not
valid).

Another important example is represented by the material implication.
Classically $P \to Q$ is equivalent to $\lnot P \lor Q$, whereas intuitionistically
$\lnot P \lor Q$ only entails (i.e., logically ``leads to'') $P \to Q$.
As we will see in~\ref{sec:formalization}, this is the reason why hypothetical reasoning
cannot be formalized using classical logic.

The semantic of intuitionistic logic is often studied relying on model-theory
using Heyting algebras or, alternatively, Kripke semantics.

Given that the main purpose of this document is to present
intuitionistic logic as a descriptive program language, a detailed analysis of the semantic
of this type of logic  as well as its syntax, will not be treated. For a depth and complete analysis please refer
to~\cite{Kripke1963-KRISAO-2} for the logic semantic and to~\cite{Dirk} for the logic syntax.

Intuitionistic logic is helpful because it is practical. In simple words, if there is
a proof that an object exists, such a proof can be used to build up an algorithm to
generate an instance of such an object.

\section{Formalization}
\label{sec:formalization}
Let's now see how to formalize hypothetical queries and hypothetical rules.
Particularly the way used by Bonner in~\cite{Bonner88alogic} will be showed.

An example of hypothetical query has already been given at the beginning of this Section.
In that example we asked the following question (or query): ``Would Phil be eligible to
be added on the wall of fame if he worked on Icarus?''. This query
can be represented by the following expression:
\begin{equation}
  \phi = wall\_of\_fame(phil) \leftarrow worked\_on(phil, icarus)
\end{equation}

We have already said that the query is satisfied (i.e., returns True) if by
adding the fact $worked\_on(phil, icarus)$ to the rulebase,  $wall\_of\_fame(phil)$ could
be inferred. Formally:

\begin{equation}
  R \cup \{worked\_on(phil, icarus)\} \vdash wall\_of\_fame(phil)
\end{equation}
Where R is the rulebase.

Therefore:
\begin{equation}
  \label{add_fact}
  R \vdash \phi\ if\   R \cup \{worked\_on(phil, icarus)\} \vdash wall\_of\_fame(phil)
\end{equation}
Which informally means: ``the expression $\phi$ is true in the rulebase R if
adding the fact $\{worked\_on(phil, icarus)\}$ among the facts in R, would cause wall\_of\_fame(phil)
to be true''.

On the other hand, hypothetical rules can be formalized in the following way:
\begin{equation}
  A \leftarrow (B \leftarrow C)
\end{equation}
which informally means: ``A is true if adding C to the rulebase would cause B to be true''.

While hypothetical queries are easier to imagine and to understand, hypothetical rules may be not.
Let's see an example of them. Let's assume that the chief of the same software company
wants to enforce the following rule: ``an employee is eligible for a promotion if
he worked on a project which made him appear on the wall of fame''. This
hypothetical rule can be expressed in the following way:

\begin{equation}
  \label{add_rule}
  promotion(e) \leftarrow \exists p [wall\_of\_fame(e) \leftarrow worked\_on(e, p)]
\end{equation}

For a deeper study of hypothetical rules and their semantic please refer to Appendix~\ref{app:hyr}.

As will be stated in Section~\ref{sec:ilp}, in order to implement hypothetical reasoning
the rulebase is temporarily modified, by adding a certain fact, such as $worked\_on(phil, icarus)$
in example~\ref{add_fact} and $worked\_on(marcus, andromeda)$ in example~\ref{add_rule}.
Once the research is completed and the goal is satisfied (or not) the
fact is removed from the rulebase, restoring it with its original content.

\subsection{A logic for hypothetical reasoning}
\label{sec:alfhr}
Hypothetical reasoning can be formalized in intuitionistic logic (as proven by
Bonner in~\cite{Bonner88alogic}), but not in classical logic. This is due to the
classical material implication. This means that the rules which govern hypothetical
reasoning are intuitionistic. In other words hypothetical reasoning has an
\textit{intuitionistic semantic}.

One question may come naturally: is there
a way to express hypothetical reasoning in classical logic?
As it has been proved by Bonner in~\cite{Bonner88alogic}, there is not.

This is due to the semantic of classical material implication. To better understand why
let's consider the following query:
``if one of $A_1$ or $A_2$ would be added
to the rulebase R, would B be true?''. To represent this query we need to construct
an expression $\phi$ such that:

\begin{equation}
  R \vdash \phi\ if\ R \cup \{A_1\} \vdash B\ or\ R \cup \{A_2\} \vdash B
\end{equation}

Such logic expression is $\phi = (B \leftarrow A_1) \lor (B \leftarrow A_2)$.
While in intuitionistic logic $\phi$ represents exactly the query
we are considering, classically it does not.

In fact, by classically elaborating the formula, we have:
\begin{equation}
  \begin{split}
  \phi &\equiv (B \leftarrow A_1) \lor (B \leftarrow A_2)\\
  &\equiv (B \lor \sim A_1) \lor (B \lor \sim A_2)\\
  &\equiv B \lor \sim A_1 \lor \sim A_2 \\
  &\equiv B \leftarrow A_1, A_2
  \end{split}
\end{equation}

Therefore:
\begin{equation}
  R \vdash \phi \iff R \cup {A_1, A_2} \vdash B
\end{equation}
Which obviously is not what we were looking for.

\subsection{Rules for hypothetical inference}
\label{sec:rfhi}
As the title of this Section itself suggests, inference rules are rules
which define what expressions can be logically defined by a set of other expressions.

Such study has been extensively carried on by several researches
in~\cite{Gabbay1984319}\cite{Miller:1989:LAM:65822.65826} and~\cite{McCarty:1988:CIL:49484.49485}.
In this document is going to be presented a simplified
version of them, as showed in~\cite{Bonner88alogic}.

Assuming R is a set of embedded implications, if B and $B_i$ are atomic, and $\phi_i$
are horn clauses, then:
\begin{equation}
  \begin{split}
    &R \vdash B\ if\ B \in R \\
    &R \vdash B\ if\ \exists a\ rule\ B \leftarrow \phi_1,...,\phi_k\ in\ R, and\ R \vdash \phi_i \forall i\\
    &R \vdash B \leftarrow B_1,..., B_k\ if\ R \cup \{B_1,...,B_k\} \vdash B
  \end{split}
\end{equation}


As mentioned in Section~\ref{sec:alfhr}, these rules have been shown in~\cite{Bonner88alogic}
to have an intuitionistic semantics, that is to say they ``make sense'' in intuitionistic logic.

Finally, a consideration about hypothetical reasoning and pure classic logic programming
(LP) has to be made. Note that in order to implement hypothetical reasoning the rulebase
has to be modified. Particularly, as it has been said in Section~\ref{sec:formalization} and Section~\ref{sec:ilp},
a rulebase is modified twice: when a fact is added
in order to infer a goal, and when the fact is removed in order to restore the content of the rulebase.
This act of modifying a rulebase is not allowed in pure, classical LP. Nonetheless, there exits
several projects which by implementing extralogical (i.e., relying on ``something'' that
is semantically outside the logic that is been employed)
extensions of LP, allow to express hypothetical reasoning.
