\begin{appendices}
  \section{Proof and Model theory}
  Intuitionistic logic, like most logics, can be defined in two formal ways:
  model-theoretically or proof-theoretically.
  These approaches merely are two ways of formalizing systems so that
  their behavior can be studied.

  Particularly, proof theory is a branch of mathematical logic that represents proofs (i.e.,
  deductive argument for a mathematical statement) as formal
  mathematical objects, so to apply mathematical techniques in
  order to study them and to derive new conclusions.

  In a programming logic context, a proof can be seen as a program (as stated by~\cite{PropAsTypes}).
  This relationship is known as Curry-Howard correspondence. Since there exists
  such a direct relationship between logic programs and proofs, it is
  natural that proof theory can be employed to model and study a logic programming
  language. Specifically, in this context,
  proof-theory is used to define the rules of execution and how they work.

  Summarizing, we can say that a proof-theoretic approach shows in details how certain
  properties of a system are satisfied.

  Another way to model and study the meaning (i.e., the semantic) of a logic is through model
  theory. As briefly mentioned in this very document, the intuitionistic semantic can be
  modeled by using the Kripke model. A mathematical model merely is the description of
  a system using mathematical concepts and language. This means that logics
  (as for example intuitionistic logic) can be studied by formalizing them as
  mathematical systems.

  Differently from a proof-theoretic perspective, a model describes a system from a
  higher point of view. In a logic programming context this means that it describes
  programs in terms of their inputs (which can be abstract), their outputs
  and (generally) a set of internal states, which in a programming language context
  represent different rulebases of the system.

  Given a logic system (e.g., an intuitionistic logic interpreter), properties and
  arguments can be proved by modeling such a system employing a proof-theoretic approach
  as well as a model-theoretic approach.
  There is not a ``right'' or ``wrong'' approach: according on what properties we want to prove
  an approach may merely be more complex and difficult to use than the other.

  \section{Hypothetical rules}
  \label{app:hyr}
  As discussed in Section~\ref{sec:formalization}, hypothetical rules can be defined
  using the form:
  \begin{equation}
    A \leftarrow (B \leftarrow C)
  \end{equation}
  Which informally means: ``A is true if adding C to the rulebase would cause B to be true''.
  For completeness, a couple of considerations have to be made.
  In literature we can usually find example where A, B and C are closed formulae
  (i.e., formulae with values filled in like $worked\_on(marcus, andromeda)$). This
  is not the only way to express hypothetical rules.
  In fact, as studied by Bonner in~\cite{Bonner88alogic} we can also express
  hypothetical rules using free variables, such as:
  \begin{equation}
    promotion(e) \leftarrow \exists p [wall\_of\_fame(e) \leftarrow worked\_on(e, p)]
  \end{equation}
  Which informally means: ``if we can find a person and a project he worked on such that the
  predicate $wall\_of\_fame$ applied on such a person is true, then such a person
  is eligible for a promotion''. The resolution procedure is the same as the one expressed
  in Section~\ref{sec:formalization}. Particularly, $worked\_on(e,p)$ is added
  to the rulebase and a proof for the goal wall\_of\_fame(e) is searched.
  If e and p are unified with some values such that wall\_of\_fame(e) becomes true, then
  promotion(e) is proven to be true.

  A second consideration involves nested hypothetical rules. The question arises naturally:
  are rules like $A \leftarrow (B \leftarrow (C \leftarrow D)))$ allowed? What
  is the semantic of such rules? Applying the same reasoning carried out so far,
  such a rule would informally mean ``A is true if, B is true if adding D to the rulebase
  would cause C to be true'', which logically seems to be semantically equivalent to
  $A \leftarrow (C \leftarrow D))$.

  At the best of our knowledge noone in the literature provided an answer to this question.
  Works such as~\cite{Bonner88alogic}~\cite{Gabbay1984319}
  and~\cite{Bonner94hypotheticalreasoning} only treat hypothetical rules in the first
  form, that is to say $A \leftarrow (B \leftarrow C)$. The personal idea of the author
  of this work, as just stated, is that the semantic of a nested hypothetical rule
  can be reduced to the contracted form $A \leftarrow (B \leftarrow C)$.
\end{appendices}
